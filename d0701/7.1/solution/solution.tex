\documentclass[11pt, a4paper]{article}
\usepackage[slantfont, boldfont]{xeCJK}
\usepackage{ulem}
\usepackage{amsmath}
\usepackage{booktabs}
\usepackage{colortbl}
\usepackage{indentfirst}
\usepackage[top = 1.0in, bottom = 1.0in, left = 1.0in, right = 1.0in]{geometry}

\setCJKmainfont{SimSun}
\setCJKmonofont{SimSun}

\setlength{\parskip}{0.8\baselineskip}
\setlength{\parindent}{2em}

\newcolumntype{Y}{>{\columncolor{red}}p{12pt}}
\newcolumntype{N}{>{\columncolor{white}}p{12pt}}


\title{某AFO狗のNOI模拟题题解}
\author{by jiry\_2}


\begin{document}
\maketitle

\begin{center}
\emph{竞赛时长:5小时}
\end{center}
\begin{center}
\begin{tabular}{|l|p{70pt}|p{70pt}|p{70pt}|}
	\hline
	题目名称 & 数 & 树 & 数据结构\\
	\hline
	输入文件名 & number.in & tree.in & data.in \\
	\hline
	输出文件名 & number.out & tree.out & data.out \\
	\hline
	每个测试点时限 & $1$s & $2.5$s & $2.5$s\\
	\hline
	测试点数目 & $10$ & $10$ & $10$\\
	\hline
	每个测试点分值 & $10$ & $10$ & $10$\\
	\hline
	内存限制 & $512$MB & $512$MB & $512$MB\\
	\hline
	是否有部分分 & 否 & 否 & 否\\
	\hline
	题目类型 & 传统 & 传统 & 传统\\
	\hline
	是否有SPJ & 否&否 &否 \\
     \hline
\end{tabular}
\end{center}

\begin{center}
\end{center}

\newpage
\section{数}
因为 $10^K$ 一定是 $2^K$ 的倍数,所以给一个正数加上 $10^K$ 后,它的二进制表示的后 $K$ 位一定是不变的。

令 $S_k$ 为所有满足 $x < 10^k$ 且 $x$ 的十进制表示的后 $k$ 位于二进制表示的后 $k$ 位相同(不足补 $0$) 的数字 $x$ 的集合。我们可以简单地从 $S_k$ 扩展出 $S_{k+1}$,只要枚举第 $k+1$ 位是 $0$ 还是 $1$ 然后判断二进制表示的第 $k+1$ 位是否相同即可。

要支持的操作有给一个数加上 $10^k$,询问一个数二进制表示下的第 $K$ 为值。可以在 $2^a$ 进制下用高精度存储 $S_k$ 中的数,预处理出 $10^k$ 在这个精度下的表示,这样就能做到第一个操作 $O(\frac{L}{a})$,第二个操作 $O(1)$,$L$ 是数字长度。

这样的时间复杂度大约是 $O(\frac{nL}{a})$ 的,标程取了 $a=25$。

\section{树}
因为树是二分图,所以树的最大独立集大小等于节点数减去最大匹配。

求树的最大匹配是一个经典的 DP 问题,令 $f[i]$ 为第 $i$ 个点的子树中选了第 $i$ 个点的最大匹配数,$g[i]$ 为不选第 $i$ 个点的最大匹配数。

因为这题要求计数,所以可以记 DP 值为状态(就是 DP 套 DP)。直接压的话状态数是 $O(n^3)$ 的,有点难以接受。不过不难发现在求最大匹配的 DP 过程中,DP 值一定满足 $0 \leq f[i]-g[i] \leq 1$,所以可以令 $F[i][j]$ 为 $i+1$ 个点,根节点的 $g$ 值为 $j$,$f$ 值为 $j+1$ 的树的个数,$G[i][j]$ 为 $i+1$ 个点,根节点的 $g$ 值为 $j$,$f$ 值为 $j$ 的树的个数。考虑每一次给根节点接上一个新的子树,就能得到转移方程,具体如下:
\begin{align*}
G[i][j]&=\sum_{a=1}^i\sum_{b=0}^j G[i-a][j-b] \times F[a-1][b-1]+[i=0,j=0]\\
F[i][j]&=\sum_{a=1}^i\sum_{b=0}^j F[i-a][j-b] \times (G[a-1][b]+F[a-1][b-1])+G[i-a][j-b] \times G[a-1][b]
\end{align*}

直接动态规划的时间复杂度是 $O(n^4)$。我们可以用 FFT 来优化第二重循环,这样时间复杂度就变成了 $O(n^3 \log n)$。

更进一步地,在 DP 时我们可以直接使用 DFT 后的点值进行运算,在最后用 $n$ 次 IDFT 还原出答案数组,时间复杂度 $O(n^3+n^2 \log n)$,可以通过所有测试点。

实际上存在复杂度更低的算法,但是因为常数比较大,所以难以在合理的时限范围内与 $O(n^3)$ 算法产生区分(好像一个点开一分钟的时限还是能卡掉的)。

可以将 $G$ 和 $F$ 看成关于节点数 $x$,根节点的 $g$ 值 $y$ 的形式幂级数,可以得到方程:
\begin{align*}
G&=xyFG+1\\
F&=xFG+xyF^2+xG^2
\end{align*}

二元形式幂级数比较难处理,但是可以发现 $y$ 的值永远不会超过 $\lfloor \frac{n}{2} \rfloor$,所以可以令 $x=y^{\lfloor \frac{n}{2} \rfloor+1}$,这样就变成了一元的问题。

有了这个方程后可以直接使用分治 FFT 的方法求出两个形式幂级数的前若干项系数。时间复杂度 $O(n^2 \log^2 n)$。

当然也可以选择解这个方程,带入后可以得到:
\begin{align*}
\frac{xG(G-1)}{xyG}+xy(\frac{G-1}{xyG})^2+xG^2-\frac{G-1}{xyG}=0
\end{align*}

转化成一维问题后牛顿迭代一波就能得到答案了,时间复杂度 $O(n^2 \log n)$。
\section{数据结构}

考虑如何计算一个序列的权值。我们让每一个子序列都在其第一次出现的位置(下标字典序最小的出现位置)被统计,这样就不用去重了。令 $dp[i]$ 为在第 $i$ 位被统计的子序列个数,令 $ne[i][j]$ 为第 $i$ 个位置后第一次出现数字 $j$ 的下标,那么 $dp[i]$ 可以更新到所有 $dp[ne[i][j]]$。这个字符串的权值为 $\sum_{i=1}^n dp[i]$。

这样暴力 DP 单组询问的时间复杂度是 $O(nK)$,考虑用数据结构优化这个这个 DP。对序列建线段树,每一个节点 $i$ 记录 $f[i][u][v]$ 表示这个区间中开头为 $v$ 的子序列,更新到区间后第一个 $u$ 的和。那么合并两个区间就相当于进行一次矩阵乘法。

在给区间加上一个数时,区间中任意两个数的相等关系并不会变,因此只需要给矩阵的行列置换一下就好了,内容本质上并没有变化。

在给区间乘上一个数 $x$ 时,如果 $(x,K)=1$,相等关系同样不会变化,处理方法和区间加一样。对于其他情况,区间中的数以模 $\frac{K}{(x,K)}$ 的余数划分成了 $\frac{K}{(x,K)}$ 个等价类,这样就不能直接用原有的值更新了。处理方法是我们对 $K$ 的所有约数 $d$,都用一棵线段树维护模 $d$ 时的答案,在区间乘法时相当于用模 $\frac{d}{(x,d)}$ 的信息去更新模 $d$ 的信息。

时间复杂度 $O(nK^3\log n)$


\end{document}

